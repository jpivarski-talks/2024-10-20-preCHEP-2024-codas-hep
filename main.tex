\pdfminorversion=4
\documentclass[aspectratio=169]{beamer}

\mode<presentation>
{
  \usetheme{default}
  \usecolortheme{default}
  \usefonttheme{default}
  \setbeamertemplate{navigation symbols}{}
  \setbeamertemplate{caption}[numbered]
  \setbeamertemplate{footline}[frame number]  % or "page number"
  \setbeamercolor{frametitle}{fg=white}
  \setbeamercolor{footline}{fg=black}
} 

\usepackage[english]{babel}
\usepackage{inputenc}
\usepackage{tikz}
\usepackage{courier}
\usepackage{array}
\usepackage{bold-extra}
\usepackage{minted}
\usepackage[thicklines]{cancel}
\usepackage{fancyvrb}

\xdefinecolor{dianablue}{rgb}{0.18,0.24,0.31}
\xdefinecolor{darkblue}{rgb}{0.1,0.1,0.7}
\xdefinecolor{darkgreen}{rgb}{0,0.5,0}
\xdefinecolor{darkgrey}{rgb}{0.35,0.35,0.35}
\xdefinecolor{darkorange}{rgb}{0.8,0.5,0}
\xdefinecolor{darkred}{rgb}{0.7,0,0}
\definecolor{darkgreen}{rgb}{0,0.6,0}
\definecolor{mauve}{rgb}{0.58,0,0.82}

\title[2024-10-20-preCHEP-2024-codas-hep]{CoDaS-HEP, US-CMS, US-ATLAS}
\author{Jim Pivarski}
\institute{Princeton University -- IRIS-HEP}
\date{October 10, 2024}

\usetikzlibrary{shapes.callouts}

\usepackage{array}
\newcolumntype{L}[1]{>{\raggedright\let\newline\\\arraybackslash\hspace{0pt}}m{#1}}
\newcolumntype{C}[1]{>{\centering\let\newline\\\arraybackslash\hspace{0pt}}m{#1}}

\begin{document}

\logo{\pgfputat{\pgfxy(0.11, 7.4)}{\pgfbox[right,base]{\tikz{\filldraw[fill=dianablue, draw=none] (0 cm, 0 cm) rectangle (50 cm, 1 cm);}\mbox{\hspace{-8 cm}\includegraphics[height=1 cm]{princeton-logo-long.png}\hspace{0.1 cm}\raisebox{0.1 cm}{\includegraphics[height=0.8 cm]{iris-hep-logo-long.png}}\hspace{0.1 cm}}}}}

\begin{frame}
  \titlepage
\end{frame}

\logo{\pgfputat{\pgfxy(0.11, 7.4)}{\pgfbox[right,base]{\tikz{\filldraw[fill=dianablue, draw=none] (0 cm, 0 cm) rectangle (50 cm, 1 cm);}\mbox{\hspace{-8 cm}\includegraphics[height=1 cm]{princeton-logo.png}\hspace{0.1 cm}\raisebox{0.1 cm}{\includegraphics[height=0.8 cm]{iris-hep-logo.png}}\hspace{0.1 cm}}}}}

% Uncomment these lines for an automatically generated outline.
%\begin{frame}{Outline}
%  \tableofcontents
%\end{frame}

% START START START START START START START START START START START START START

\begin{frame}{Three of this year's HEP-computing tutorials in the U.S.}
\Large
\vspace{0.5 cm}
\begin{columns}
\column{1.08\linewidth}
\begin{itemize}\setlength{\itemsep}{0.5 cm}
\item June 20--21 (2 days): US-CMS at Princeton

\textcolor{gray}{\normalsize Alexander Held, Andrzej Novak, Elliott Kauffman, Jim Pivarski, Lindsey Gray, \\ Matthew Feickert, Nick Manganelli, Nick Smith, Oksana Shadura, Peter Elmer}

\item July 18--19 (2 days): US-ATLAS at U.\ Washington

\textcolor{gray}{\normalsize Alexander Held, Ana Peixoto, Fengping Hu, Gordon Watts, Jim Pivarski, \\ Kyungeon Choi, Lindsey Gray, Matthew Feickert, Oksana Shadura, Vangelis Kourlitis}

\item July 22--26 (5 days): CoDaS-HEP at Princeton

\textcolor{gray}{\normalsize Andres Rios-Tascon, David Lange, Henry Schreiner, Ianna Osborne, Jim Pivarski, Kilian Lieret, Louis-Guillaume Gagnon, Peter Elmer, Steve Lantz, Sudhir Malik, Tim Mattson}
\end{itemize}
\end{columns}
\end{frame}

\begin{frame}{Photos (from CoDaS-HEP)}
\vspace{0.5 cm}
\begin{columns}
\column{0.33\linewidth}
\includegraphics[width=\linewidth]{PHOTOS/DSCF2628.jpg}

\column{0.33\linewidth}
\includegraphics[width=\linewidth]{PHOTOS/DSCF2637.jpg}

\column{0.33\linewidth}
\includegraphics[width=\linewidth]{PHOTOS/DSCF2962.jpg}

%% \column{0.25\linewidth}
%% \includegraphics[width=\linewidth]{PHOTOS/DSCF3047.jpg}
\end{columns}

\begin{columns}
\column{0.33\linewidth}
\includegraphics[width=\linewidth]{PHOTOS/DSCF2928.jpg}

\column{0.33\linewidth}
\includegraphics[width=\linewidth]{PHOTOS/DSCF2805.jpg}

\column{0.33\linewidth}
\includegraphics[width=\linewidth]{PHOTOS/DSCF2642.jpg}
\end{columns}
\end{frame}

\begin{frame}{\mbox{ }}
\LARGE
\begin{center}
\textcolor{darkblue}{Content/teaching styles/technologies}
\end{center}
\end{frame}

\begin{frame}{All three events consisted of}
\Large
\vspace{0.5 cm}
\begin{itemize}\setlength{\itemsep}{0.25 cm}
\item<1-> Lecture-style presentations (PDF, PowerPoint, Keynote)
\item<2-> Lectures mixed with small problems (Jupyter)
\item<3-> Long exercises: from 20 minutes to 2 hours
\item<4-> Catered breakfasts and lunches, coffee breaks
\item<5-> Social dinners and student bonding in dorms, pub crawls\ldots
\end{itemize}

\vspace{0.5 cm}
\large
\uncover<6->{Considerable sharing of teaching materials between events (including HSF-India), and from one year to the next.}

\vspace{0.25 cm}
\uncover<7->{With one exception, none of this material was from \textcolor{blue}{\url{hsf-training.org}}.}
\end{frame}

\begin{frame}{Interactive exercises 1 (me)}
\vspace{0.25 cm}
\large
\begin{columns}
\column{0.8\linewidth}
\includegraphics[width=\linewidth]{PLOTS/teacher-student-notebook-pair.png}

\column{0.3\linewidth}
\textcolor{darkorange}{\bf Columnar analysis}

\tiny
\vspace{0.2 cm}
\textcolor{blue}{\href{https://github.com/jpivarski-talks/2024-06-20-uscms-princeton-tutorial}{https://github.com/jpivarski-talks/2024-06-20-uscms-princeton-tutorial}}

\textcolor{blue}{\href{https://github.com/jpivarski-talks/2024-07-18-usatlas-seattle-tutorial}{https://github.com/jpivarski-talks/2024-07-18-usatlas-seattle-tutorial}}

\small
\vspace{0.2 cm}
\uncover<2->{Consisted almost entirely of 5--10 minute puzzles.}

\vspace{0.2 cm}
\uncover<3->{{\bf Two notebooks:} teacher.ipynb has more background, student.ipynb just sets up the problems.}

\vspace{0.2 cm}
\uncover<4->{Students \mintinline{python}{send_answer} anonymously to the teacher notebook, where we review.}

\vspace{0.2 cm}
\uncover<5->{I don't like how I had to set this up (Amazon SNS): it was too complicated.}
\end{columns}
\end{frame}

\begin{frame}{Interactive exercises 2 (Nick Smith, Nick Manganelli, Alex Held)}
\vspace{0.25 cm}
\large
\begin{columns}
\column{0.73\linewidth}
\includegraphics[width=\linewidth]{PLOTS/sample-guessing-game.png}

\column{0.3\linewidth}
\textcolor{darkorange}{\bf Sample game}

\tiny
\vspace{0.2 cm}
\textcolor{blue}{\href{https://github.com/iris-hep/uscms-idap-training/blob/main/coffea/coffea-challenge-samplegame.ipynb}{https://github.com/iris-hep/uscms-idap-training/blob/main/coffea/coffea-challenge-samplegame.ipynb}}

\small
\vspace{0.225 cm}
\uncover<2->{1.5 hours (+ overnight)}

\vspace{0.225 cm}
\uncover<3->{Given 6 physics samples, students use any tools necessary to figure out which was generated by which physics process.}

\vspace{0.225 cm}
\uncover<4->{Naturally, this tests both computing {\it and} physics knowledge.}

\vspace{0.225 cm}
\uncover<5->{Results are submitted in a Google Form.}
\end{columns}
\end{frame}

\begin{frame}{Interactive exercises 3 (Kilian Lieret)}
\vspace{0.25 cm}
\large
\begin{columns}
\column{0.72\linewidth}
\includegraphics[width=\linewidth]{PLOTS/github-playground.png}

\column{0.3\linewidth}
\textcolor{darkorange}{\bf Git(Hub) playground}

\tiny
\vspace{0.2 cm}
\textcolor{blue}{\href{https://github.com/klieret/collab-git-playground-codas-hep-24}{https://github.com/klieret/collab-git-playground-codas-hep-24}}

\small
\vspace{0.35 cm}
\uncover<2->{1.5 hours of mixed lecture and exercises}

\vspace{0.35 cm}
\uncover<3->{Students fork, branch, open pull requests, handle merge conflicts, etc.\ in a single git repo, {\it all at the same time.}}

\vspace{0.35 cm}
\uncover<4->{The chaos that ensues is part of the learning process---this can {\it only} be done in a large group.}

\end{columns}
\end{frame}

\begin{frame}{Interactive exercises 4 (Tim Mattson)}
\vspace{0.25 cm}
\large
\begin{columns}
\column{0.72\linewidth}
\includegraphics[width=\linewidth]{PLOTS/parallel-processing.png}

\column{0.3\linewidth}
\vspace{0.2 cm}

\textcolor{darkorange}{\bf Parallel programming}

\tiny
\vspace{0.2 cm}
\textcolor{blue}{\href{https://indico.cern.ch/event/1422680/contributions/5983265/attachments/2900081/5085486/intro_par_prog_with_Openmp.pdf}{https://indico.cern.ch/event/1422680/ \\
%
contributions/5983265/attachments/2900081/
%
5085486/intro\_par\_prog\_with\_Openmp.pdf}}

\small
\vspace{0.2 cm}
\uncover<2->{1/2 hour per problem}

\vspace{0.2 cm}
\uncover<3->{Students copy serial programs, compile them, and parallelize them.}

\vspace{0.2 cm}
\uncover<4->{Making students type whole programs manually is a good learning experience!}

\vspace{0.2 cm}
\uncover<5->{Can use editor + terminal of JupyterLab if necessary; most students used their laptops directly.}

\end{columns}
\end{frame}

\begin{frame}{Interactive exercises 5 (Yana Osborne and me)}
\vspace{0.25 cm}
\large
\begin{columns}
\column{0.73\linewidth}
\includegraphics[width=\linewidth]{PLOTS/long-form-problem-sets.png}

\column{0.3\linewidth}
\textcolor{darkorange}{\bf Columnar analysis}

\tiny
\vspace{0.2 cm}
\textcolor{blue}{\href{https://github.com/ianna/2024-07-24-codas-hep-columnar-data-analysis}{https://github.com/ianna/2024-07-24-codas-hep-columnar-data-analysis}}

\small
\vspace{0.25 cm}
\uncover<2->{Each lesson has lecture with short problems in slides (jupyterlab-deck) and a workbook (Jupyter), like the teacher/student notebooks, but also has two long (half hour) problems.}

\small
\vspace{0.25 cm}
\uncover<3->{Slides + workbook is done together, but long problems are on their own/in groups.}

\small
\vspace{0.25 cm}
\uncover<4->{Since there's a choice of problems, it's hard to present solutions.}
\end{columns}
\end{frame}

\begin{frame}{Interactive exercises 6 (me)}
\vspace{0.25 cm}
\large
\begin{columns}
\column{0.72\linewidth}
\includegraphics[width=\linewidth]{PLOTS/ml-results-in-google-sheet.png}

\column{0.3\linewidth}
\textcolor{darkorange}{\bf Machine learning}

\tiny
\vspace{0.2 cm}
\textcolor{blue}{\href{https://github.com/jpivarski-talks/2024-07-24-codas-hep-ml}{https://github.com/jpivarski-talks/2024-07-24-codas-hep-ml}}

\small
\vspace{0.25 cm}
\uncover<2->{After a lecture with small problems, students had to build a neural network from scratch in 2 hours (data and problem given).}

\small
\vspace{0.25 cm}
\uncover<3->{Results were collected in a shared Google spreadsheet: they pasted ROC curve results into a column and all results were plotted.}

\small
\vspace{0.25 cm}
\uncover<4->{(Much easier to set up than \mintinline{python}{send_answer}.)}
\end{columns}
\end{frame}

\begin{frame}{General comments/conclusions}
\large
\vspace{0.5 cm}
\begin{itemize}\setlength{\itemsep}{0.45 cm}
\item<1-> We know what we want to teach; it's easy to write talks/lectures on these topics but hard to create interactive problems at the right level: we informally reuse them by copy-paste-and-edit.
\item<2-> \textcolor{blue}{\url{hsf-training.org}} would be more useful to us as a repository of problem sets that we can mix into our lessons, rather than full lessons.
\item<3-> We also want to feed student solutions back into the main lecture, to discuss them, and we have been trying different technologies to do that.
\begin{itemize}
\item It's easier with an off-the-shelf product, like Google Sheets, Forms, and GitHub.
\end{itemize}
\item<4-> There are reasons to have both
\begin{itemize}
\item short problems to keep students engaged in a lecture, ``on rails'' to keep them short,
\item long problems to simulate real problem-solving, ``open world'' for realism.
\end{itemize}

\end{itemize}
\end{frame}

\begin{frame}{\mbox{ }}
\LARGE
\begin{center}
\textcolor{darkblue}{Getting software to students}
\end{center}
\end{frame}

\begin{frame}{This is a surprisingly hard problem}
\vspace{0.3 cm}
\begin{columns}
\column{1.1\linewidth}\setlength{\extrarowheight}{-0.15 cm}
\begin{tabular}{p{0.3\linewidth} p{0.35\linewidth} C{0.14\linewidth} C{0.12\linewidth}}
{\bf Method} & {\bf Failure modes} & {\bf P(works for everyone)} & {\bf Reusable afterward} \\\hline
\uncover<1->{Have students install everything on their own laptops; venv, conda-forge, Docker} & \uncover<1->{Windows; not having the software to install the software; mystery errors we can't spend time to solve} & \uncover<1->{\vspace{-0.4 cm}$1 - 0.9^N$} & \uncover<1->{\vspace{-0.4 cm}yes} \\
\uncover<2->{Public cloud-based Binder (\textcolor{blue}{\href{mybinder.org}{mybinder.org}})} & \uncover<2->{Stuck loading image; crashes without persistence} & \uncover<2->{\vspace{-0.4 cm}$0.8$} & \uncover<2->{\vspace{-0.4 cm}yes} \\
\uncover<3->{GitHub Codespaces} & \uncover<3->{Big images; boots in VSCode, not Jupyter (unless configured to)} & \uncover<3->{\vspace{-0.4 cm}$0.95$} & \uncover<3->{\vspace{-0.4 cm}yes} \\
\uncover<4->{Google Colab (with GPUs!)} & \uncover<4->{Persistence; not Jupyter (old fork)} & \uncover<4->{\vspace{-0.4 cm}$0.95$} & \uncover<4->{\vspace{-0.4 cm}yes} \\
\uncover<5->{CERN Swan} & \uncover<5->{CERN accounts} & \uncover<5->{\vspace{-0.4 cm}$1 - 0.8^N$} & \uncover<5->{\vspace{-0.4 cm}yes} \\
\uncover<6->{Paid cloud solution} & \uncover<6->{Authentication} & \uncover<6->{\vspace{-0.4 cm}$0.95$} & \uncover<6->{\textcolor{red}{\vspace{-0.4 cm}no}} \\
\uncover<7->{In-browser JupyterLite} & \uncover<7->{Not all packages can be used} & \uncover<7->{\vspace{-0.4 cm}\textcolor{red}{$1$}} & \uncover<7->{\vspace{-0.4 cm}yes} \\
\raggedright \uncover<8->{Self-hosted JupyterHub} & \uncover<8->{Authentication; big images; GPUs} & \uncover<8->{\vspace{-0.4 cm}$0.9$} & \uncover<8->{\vspace{-0.4 cm}maybe} \\
\end{tabular}
\end{columns}
\end{frame}

\begin{frame}{\mbox{ }}
\large
\vspace{1 cm}
More on ``Self-hosted JupyterHub/BinderHub'' in David Lange's HSF-India talk.
\end{frame}

\begin{frame}{\mbox{ }}
\LARGE
\begin{center}
\textcolor{darkblue}{Feedback from students}
\end{center}
\end{frame}

\begin{frame}{Student surveys}
\Large
\vspace{0.5 cm}
CoDaS-HEP was held in 2017, 2018, 2019, 2022, 2023, and 2024.

\vspace{0.5 cm}
\uncover<2->{I could find survey results from all but 2019.}

\vspace{0.5 cm}
\uncover<3->{Survey consists of quantitative rankings and qualitative requests for comments, some general and some for each teacher/session.}

\vspace{0.5 cm}
\uncover<4->{Per-teacher questions are useful for improving the program, but we'll only look at general questions here.}
\end{frame}

\begin{frame}{General quantitative questions}




\end{frame}




\end{document}
